\documentclass[a4paper,english]{article}

\usepackage{fullpage}

%%%%%%%%%%%%%%%%%%%%%%%% TEX FONTS AND LANGUAGE
% This is the new Latin Modern fonts (by Knuth)
\usepackage{lmodern}
% All the above merge well (IMHO) with the Euler family of
% mathematical fonts (by H. Zapf)
\usepackage{amsfonts}
%to best use the above, you have to switch to T1
\usepackage[T1]{fontenc}
% (Note that LaTeX still defaults to OT1, to be compatible with
%  old files; but nowadays you definitively want to use T1.)
\usepackage[utf8]{inputenc}
\usepackage{babel}



\usepackage{amsmath,amssymb,amsthm}


\usepackage{graphicx,color,hyperref}

\begin{document}

\begin{center}\Large
  Non-destructive encoder flushing for ``arithmetic coding''\\
  Andrea C. G. Mennucci
  \footnote{Scuola Normale Superiore, Pisa, Italy}
  \\
  \today
\end{center}

\section{Introduction}

In this short essay we discuss \emph{arithmetic coding}, as described in
\cite{witten1987arithmetic}.
(The reference implementation is the code in \texttt{ArithCodeTut}
\footnote{\url{https://github.com/mennucc/ArithCodeTut.git}} in \texttt{github.com}).

In particular we discuss the problem of \emph{flushing}: a method implemented
in the encoder that makes sure that the decoder has received all previously
encoded symbols.

In \cite{witten1987arithmetic}, in the section \emph{``Termination''}, the authors write
\footnote{Rephrased for clarity}:
\emph{To finish the transmission, it is necessary to send
  [\ldots]  enough bits to ensure that the encoded string falls
  within the finale range. [\ldots] The encoder  need only transmit 01 if
\texttt{Low} < \texttt{First\_qtr}, or 10 otherwise.}



\smallskip

While this approach  works, it has a defect: after that sequence
is emitted, it is unclear how the encoder could encode more symbols
and send more bits to the decoder.


This may property may be of interest in some cases: indeed the Decoder
can have an arbitrarily large delay in decoding the symbols
that the encoder has already seen.
(See next section). So a periodic flushing may be desirable
--- as long as encoding/decoding can proceed after flushing.

\medskip

In the following a \emph{Bernoulli symbol} is a symbol that has two choices, such as $0,1$ or
$a,b$, each choice with probability 1/2 (unless otherwise stated).

\medskip

The questions now is: can we justify the flushing output of the
encoder as result of an input, let's say as iid Bernoulli symbols?





\section{Example}

We model the encoder as follows.

Let $[\alpha_j,\beta_j)$ the symbol interval
stored in the Encoder after inserting  the j-th symbol using \texttt{input\_symbol()}:
the encoder is in \emph{dirty state}.
Let $[\tilde \alpha_j,\tilde \beta_j]$
the symbol interval after the Encoder has emitted all bits (either by callback, or by polling),
so the Encoder is in \emph{clean state}: at this point
either
\[ 0\le \tilde \alpha_j< 1/4 \land 1/2\le \tilde \beta_j\le 1\]
(condition (1a) in \cite{witten1987arithmetic})
or
\[0\le \tilde \alpha_j< 1/2 \land 3/4\le\tilde \beta_j\le 1\]
(condition (1b) in \cite{witten1987arithmetic})
%(up to some equalities)

Now suppose this happens.

The encoder receives many symbols, say $N$, but it does not emit any bit,
neither store any virtual bits in \texttt{bits\_to\_follow},
because
\[ 0\le \alpha_j< 1/4\land 1/2\le \beta_j\le 1\] at all $j\le N$;
so $\tilde\alpha_j=\alpha_j$, $\tilde\beta_j=\beta_j$.
(A symmetric situation is possible)

At that time the decoder is still at the pristine state $[0,1]$.

\smallskip

We suppose that the decoder knows that it will be flushed after receiving N symbols;
either because this number was passed in a header of the encoded
file, or because the N-th symbol is a special \texttt{WILL\_FLUSH} symbol
(similarly to the \texttt{EOF} symbol in  \cite{witten1987arithmetic}).

\smallskip

The encoder is then flushed, so that it emits enough bits so that the
decoder can decode the N symbols. Since $\alpha_N< 1/4$, then the
encoder sends a 0 and then a 1 (and would send
\texttt{bits\_to\_follow} other ones, but in this example there are no
virtual bits, for simplicity).

The decoder bit interval is now
\[ [1/4, \, 1/2) = [ 0.01000\ldots 0 , \, 0.0111\ldots 1] \]
(where on the right we express the binary representation of the interval in the algorithm);
the decoder understands all symbols up to symbol $N$:
indeed there is only one string of symbols such that $[\alpha_N,\beta_N]\supseteq [1/4,\, 1/2]$

\section{Non-destructive flushing}

\subsection{Low case}
We consider the case
\[ 0\le \alpha_N< 1/4\land 1/2\le \beta_N\le 1\quad.\]

We wish explain the sequence ``01'' that the Encoder has sent for flushing,
as the encoding of 3 i.i.d Bernoulli symbols.
For simplicity though we express it as the sending of one symbol $s$ uniformly
distributed in $\{0,1,2,3,4,5,6,7\}$: this symbol indeed is equivalent
to ``3 input bits''
(up to numerical approximation).

So
\[\alpha_{N+1}= \alpha_N+\frac{7-s}{8}(\beta_N-\alpha_N)  \quad,\quad
  \beta_{N+1}=\alpha_N+\frac{8-s}{8}(\beta_N-\alpha_N)\]
(we follow the convention of  \cite{witten1987arithmetic}, that
subintervals are in decreasing order as $s$ increases --- this is also
the convention used in the reference code).

Since the width of $[\alpha_N,\beta_N]$ is at most 1,
then the width of any subinterval  $[\alpha_{N+1},\beta_{N+1}]$
is at most $1/8$;
so there must exist a choice of $s$ such that  $[\alpha_{N+1},\beta_{N+1}]$
is contained in 
\[ [1/4, \, 1/2) = [ 0.01000\ldots 0 , \, 0.0111\ldots 1] \]
(that has width $1/4$).

We will show that a possible choice is
\[s = \left\lfloor \frac{8 \beta_N - 2}{(\beta_N-\alpha_N) }\right\rfloor - 1 \]

Hence, if we input that symbol in the encoder after the $N$-th symbol,
we will obtain two important results:
\begin{itemize}
\item the Encoder will emit $01$, that will flush the system
  (it may emit a further bit; we skip the discussion);
\item the Decoder will be able to decode the symbol,
  and Encoding/Decoding may proceed behind the flush.
\end{itemize}

\begin{proof}
To be consistent with what the decoder is receiving
\begin{align}
  1/4\le   \alpha_{N+1} \quad,\quad \beta_{N+1}< 1/2
\end{align}
that is
\begin{align*}
  1/4\le   \alpha_{N+1}
  \iff \\
  1/4\le  \alpha_N+\frac{7-s}{8}(\beta_N-\alpha_N)
  \iff \\
  \frac{2- 8 \alpha_N}{(\beta_N-\alpha_N) } \le  7-s
  \iff \\
  \frac{2-  \alpha_N - 7 \beta_N}{(\beta_N-\alpha_N) } \le  -s
  \iff \\
  s\le  \frac{-2+  \alpha_N + 7 \beta_N}{(\beta_N-\alpha_N) } =  \frac{8 \beta_N - 2}{(\beta_N-\alpha_N) } - 1
\end{align*}
whereas
\begin{align*}
  \beta_{N+1}< 1/2
  \iff \\
  \alpha_N+\frac{8-s}{8}(\beta_N-\alpha_N) < 1/2
  \iff\\
  8 -s < \frac{{4 - 8 \alpha_N}}{(\beta_N-\alpha_N)}
  \iff \\
   -s < \frac{{4 - 8 \beta_N}}{(\beta_N-\alpha_N)}
  \iff \\
   s > \frac{{8 \beta_N - 4}}{(\beta_N-\alpha_N)}
\end{align*}
summarizing
\begin{equation}
  \frac{{8 \beta_N - 4}}{(\beta_N-\alpha_N)} <  s\le  \frac{8 \beta_N - 2}{(\beta_N-\alpha_N) } - 1\label{eq:low_s_ineq}
\end{equation}

We double-check that the interval contains an integer, indeed the difference of the extremes is
\[    \frac{2}{(\beta_N-\alpha_N) } - 1 \ge  1 \]
\end{proof}


\subsection{Examples}

Some examples of
possible values of
\[s = \left\lfloor \frac{8 \beta_N - 2}{(\beta_N-\alpha_N) }\right\rfloor - 1 \]
\begin{itemize}
\item If $\alpha_N=1/4$ then
  \[ \frac{8 \beta_N - 2}{(\beta_N-1/4) } = 8\]
  so $s=7$, regardless of $\beta_N$.


\item If $\alpha_N=0$
  \[ \frac{8 \beta_N - 2}{\beta_N } = 8 - \frac{2}{\beta_N }\]
  so
  \[s = 7 - \left\lceil \frac{2}{\beta_N }\right\rceil \quad, \]
  so $s\in\{5,6,7\}$;
\item in particular if if $\beta_N=1, \alpha_N=0$ then $s=5$.
\end{itemize}


\subsection{High case}
We consider the case
\[ 0\le \alpha_N< 1/2\land 3/4\le  \beta_N\le 1\quad.\]
In this case the encoder emits $10$; a good choice is
\[s = \left\lfloor \frac{8 \beta_N - 4}{(\beta_N-\alpha_N) }\right\rfloor - 1 \]
\begin{proof}
To be consistent with what the decoder is receiving
\begin{align}
  1/2\le   \alpha_{N+1} \quad,\quad \beta_{N+1}< 3/4
\end{align}
that is
\begin{align*}
  1/2\le   \alpha_{N+1}
  \iff \\
  1/2\le  \alpha_N+\frac{7-s}{8}(\beta_N-\alpha_N)
  \iff \\
  \frac{4- 8 \alpha_N}{(\beta_N-\alpha_N) } \le  7-s
  \iff \\
  \frac{4-  \alpha_N - 7 \beta_N}{(\beta_N-\alpha_N) } \le  -s
  \iff \\
  s\le  \frac{-4+  \alpha_N + 7 \beta_N}{(\beta_N-\alpha_N) } =  \frac{8 \beta_N - 4}{(\beta_N-\alpha_N) } - 1
\end{align*}
whereas
\begin{align*}
  \beta_{N+1}< 3/4
  \iff \\
  \alpha_N+\frac{8-s}{8}(\beta_N-\alpha_N) < 3/4
  \iff\\
  8 -s < \frac{{6 - 8 \alpha_N}}{(\beta_N-\alpha_N)}
  \iff \\
   -s < \frac{{6 - 8 \beta_N}}{(\beta_N-\alpha_N)}
  \iff \\
   s > \frac{{8 \beta_N - 6}}{(\beta_N-\alpha_N)}
\end{align*}
summarizing
\begin{equation}
  \frac{{8 \beta_N - 6}}{(\beta_N-\alpha_N)} <  s\le  \frac{8 \beta_N - 4}{(\beta_N-\alpha_N) } - 1\label{eq:high_s_ineq}
\end{equation}

We double-check that the interval contains an integer, indeed the difference of the extremes is
\[    \frac{2}{(\beta_N-\alpha_N) } - 1 \ge  1 \]
\end{proof}


\subsection{Remark on symmetry}
Suppose
\[ 0\le \hat \alpha_N< 1/2\land 3/4\le \hat \beta_N\le 1\quad.\]
This is symmetric of the ``low'' case, up to defining
\[\hat\alpha_j=1-\beta_j\quad,\quad \hat \beta_j = 1 - \alpha_j \]
by defining $\hat s=8-s$
the inequality
\eqref{eq:low_s_ineq}
becomes  the inequality
\eqref{eq:high_s_ineq};
but the role of $\le$ and $<$ is inverted.





\providecommand{\bysame}{\leavevmode\hbox to3em{\hrulefill}\thinspace}
\begin{thebibliography}{1}
\bibitem{witten1987arithmetic}
Ian~H Witten, Radford~M Neal, and John~G Cleary.
\newblock Arithmetic coding for data compression.
\newblock {\em Communications of the ACM}, 30(6):520--540, 1987.

\end{thebibliography}


\end{document}

